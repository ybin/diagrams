
\usepackage{xparse}
\usepackage{underscore}
\usepackage{tikz}
\usetikzlibrary{shapes,arrows,arrows.meta,positioning,calc,backgrounds,matrix,fit,decorations.pathreplacing}

%%% environment struct: [label][style for dummy node]{name}{width}{height}
\DeclareDocumentEnvironment{struct}{ O{} O{} m m m }
{
    \newcount\index
    \newcommand{\diaName}{#3}
    \newcommand{\diaWidth}{#4}
    \newcommand{\diaHeight}{#5}
    \newcommand{\updateIndex}{\advance\index by 1\relax}
    \newcommand{\lastId}{\diaName-\the\numexpr\index-1\relax}
    \newcommand{\id}{\diaName-\the\index}
    %% \record[style for node]{node content}
    \newcommand{\record}[2][]{
        \node[fixed node={\diaWidth}{\diaHeight},below=of \lastId, ##1](\id){##2};
        \updateIndex
    }
    %% dummy node for label(title)
    \node[draw,fixed node={\diaWidth}{0em},label={[name=\diaName-label,yshift=0.5em]\textbf{#1}},#2](\id){};\updateIndex
}
{
    %% dummy node for wrapping all nodes in this environment
    \node[fit=(\diaName-0)(\lastId),inner sep=0pt](\diaName){};
}
%%% environment struct end


%%% environment arr: [label][style for dummy node]{name}{width}{height}
\DeclareDocumentEnvironment{arr}{ O{} O{} m O{3em} O{1em} }
{
    \newcount\index
    \newcommand{\diaName}{#3}
    \newcommand{\diaWidth}{#4}
    \newcommand{\diaHeight}{#5}
    \newcommand{\updateIndex}{\advance\index by 1\relax}
    \newcommand{\lastId}{\diaName-\the\numexpr\index-1\relax}
    \newcommand{\id}{\diaName-\the\index}
    %% \element[style for node]{node content}
    \newcommand{\element}[2][]{
        \node[label=below:\tiny{\the\numexpr\index-1\relax},fixed node={\diaWidth}{\diaHeight},right=of \lastId, ##1](\id){##2};
        \updateIndex
    }
    \newcommand{\elem}[2][]{
        \node[fixed node={\diaWidth}{\diaHeight},right=of \lastId, ##1](\id){##2};
        \updateIndex
    }
    %% dummy node for label(title)
    \node[draw,fixed node={0em}{\diaHeight},label={[name=\diaName-label]180:{\bf #1}},#2](\id){};\updateIndex
}
{
    %% dummy node for wrapping all nodes in this environment
    \node[fit=(\diaName-0)(\lastId),inner sep=0pt](\diaName){};
}
%%% environment struct end


\tikzset{
    table/.style 2 args={
        draw,
        rectangle,
        inner sep=0pt,
        matrix of nodes,
        nodes in empty cells,
        nodes={
            draw,
            font=\ttfamily,
            align=center,
            text width=#1,
            outer sep=0pt,
            inner sep=0.3em,
            minimum height=1.6em
        },
        label={[align=center]90:\bf{#2}}
    },
    table/.default={10cm}{},
    right brace/.style 2 args={
        decorate,
        decoration={brace,amplitude=#1,raise=#2}
    },
    right brace/.default={8pt}{2pt},
    right note/.style={right=#1},
    right note/.default={0.3cm},
    left note/.style={left=#1},
    left note/.default={0.3cm},
    left brace/.style 2 args={
        decorate,
        decoration={brace,amplitude=#1,raise=#2,mirror}
    },
    left brace/.default={8pt}{2pt},
    right addr note/.style={right=#1},
    right addr note/.default={0.5cm},
    left addr note/.style={left=#1},
    left addr note/.default={0.5cm},
    arrow/.style={>=stealth',->},
    fixed node/.style 2 args={
        node distance=0,
        outer sep=0,
        inner sep=0,
        draw=white,
        fill=green!50,
        font=\ttfamily,
        rectangle,
        align=center,
        minimum width=#1,
        minimum height=#2
    },
    title node/.style={rectangle,inner sep=1em},
    blank cell/.style={fill=white,minimum height=1cm},
    non blank cell/.style={fill=cyan!30,minimum height=#1}
}
