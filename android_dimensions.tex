
\documentclass[a4paper,11pt]{article}


%%% fontenc
%\usepackage{fontspec,xunicode,xltxtra}
%\setmainfont{Times New Roman}
%\setsansfont{Source Sans Pro}
%\setmonofont{Source Sans Pro}

%%% xeCJK
\usepackage{xeCJK}
%\setCJKmainfont[BoldFont={Adobe Heiti Std},ItalicFont={方正隶变简体}]{Adobe Song Std}
%\setCJKsansfont[BoldFont={Adobe Heiti Std},ItalicFont={方正隶变简体}]{Adobe Song Std}
%\setCJKmonofont+[BoldFont={Adobe Heiti Std},ItalicFont={方正隶变简体}]{Adobe Song Std}
\setCJKmainfont[BoldFont={微软雅黑},ItalicFont={微软雅黑}]{微软雅黑}
\setCJKsansfont[BoldFont={微软雅黑},ItalicFont={微软雅黑}]{微软雅黑}
\setCJKmonofont[BoldFont={微软雅黑},ItalicFont={微软雅黑}]{微软雅黑}
\XeTeXlinebreaklocale "zh"
\XeTeXlinebreakskip=0pt plus 1pt minus 0.1pt

\usepackage{xcolor}
\usepackage{graphicx}

%%% get total page number
\usepackage{lastpage}

%%% customized definition
\makeatletter
\def\sybtitle#1{\def\@sybtitle{#1}}
\def\sybauthor#1{\def\@sybauthor{#1}}
\def\sybdate#1{\def\@sybdate{#1}}
\sybtitle{}
\sybauthor{}
\sybdate{}
\def\sybmaketitle{
  \begin{center}
  \vspace*{.8in}
  {\huge\bfseries\@sybtitle}
  \par
  \vspace{.8in}
  {\Large\@sybauthor}
  \par
  \vspace{.2in}
  \@sybdate
  \vspace{.5in}
  \end{center}
}
\makeatother
\setlength{\parindent}{0pt}
\renewcommand{\today}{\number\month 月 \number\day 日, ~\number\year 年}
\def\lt{\textless}
\def\gt{\textgreater}
\renewcommand\contentsname{\bfseries 目~~录}
\newcommand\bs{\texttt{\symbol{'134}}} % input backslash sign
%\newcommand\bs{\string\} % same as above definition
\long\def\cmd#1{\par\vspace{.5em}\hspace*{2em}#1\vspace{.5em}\par}
\def\cstr#1{\texttt{\string#1}} % e.g. \cstr{\latex}
\long\def\runcode#1{\par\bigskip#1\bigskip\par}
% 我不想看到那么多的underful hbox,尤其是minted环境加上背景色之后
\hbadness=10000
% 适当放宽overful hbox的限制,运行2pt的溢出
\hfuzz=2pt
\parskip=3\lineskip


%%% change background color & add frame for enumerate enviroment
\usepackage{mdframed}
\newmdenv[backgroundcolor=blue!10,linewidth=0pt]{coloredframe}
\newenvironment{coloredenumerate}{
  \begin{coloredframe}
  \begin{enumerate}
}{
  \end{enumerate}
  \end{coloredframe}
}

%%% geometry
\usepackage[includehead,includefoot,hmargin=21mm,vmargin=10.5mm,
            headsep=12pt,headheight=25pt]{geometry}
%\usepackage[includehead,includefoot,hmargin=1.2in,vmargin=1in]{geometry}

%%% fancyhdr
\usepackage{fancyhdr}
\makeatletter
\fancypagestyle{main} {
  \fancyhf{} % clear header & footer
  \fancyhead[L]{\bfseries\@sybtitle}
  \fancyhead[R]{\thepage/\pageref*{LastPage}}
  \renewcommand{\headrulewidth}{0.4pt} % header line
  \renewcommand{\footrulewidth}{0pt} % footer line
}
\fancypagestyle{header} {
  \fancyhf{} % clear header & footer
  \fancyfoot[C]{\roman{page}}
  \renewcommand{\headrulewidth}{0pt} % header line
  \renewcommand{\footrulewidth}{0pt} % footer line
}
\makeatother

\usepackage{titlesec}
\titleformat{\part}{\centering\Large\bfseries}{第\,\thepart\,部分}{1em}{}
\titleformat{\section}{\large\bfseries}{\thesection}{1em}{}
\titleformat{\subsection}{\normalsize\bfseries}{\thesubsection}{1em}{}
%\titlespacing*{章节命令}{左边距}{上文距}{下文距}[右边距]
\titlespacing*{\section}{0pt}{2\baselineskip}{\parsep}
\usepackage{titletoc}
%\titlecontents{标题层次}[左间距]{整体格式}{标题序号}{标题内容}{指引线和页码}[下间距]
%\titlecontents{subsection}[4em]{}{\contentslabel{3em}}{}{\titlerule*[1em]{$\cdot$}\contentspage}


\usepackage[colorlinks=true,linkcolor=blue,linktoc=section]{hyperref}

%%% perfect source code display
\usepackage{minted}
%\usemintedstyle{colorful}
\definecolor{srcbg}{rgb}{0.95,0.95,0.95}
\newminted{java}{linenos,tabsize=4,bgcolor=srcbg}
\newminted{xml}{linenos,tabsize=4,bgcolor=srcbg}
\newminted{cpp}{linenos,tabsize=4,bgcolor=srcbg}
\newminted{bash}{linenos,tabsize=4,bgcolor=srcbg}
\newminted{latex}{linenos,tabsize=4,bgcolor=srcbg}
\newminted{scheme}{linenos,tabsize=4,bgcolor=srcbg}
\newminted{javascript}{linenos,tabsize=4,bgcolor=srcbg}
\newminted{sql}{linenos,tabsize=4,bgcolor=srcbg}
\newminted{make}{linenos,tabsize=4,bgcolor=srcbg}

\usepackage{amsmath}



\usepackage{xparse}
\usepackage{underscore}
\usepackage{tikz}
\usetikzlibrary{shapes,arrows,arrows.meta,positioning,calc,backgrounds,matrix,fit,decorations.pathreplacing}

%%% environment struct: [label][style for dummy node]{name}{width}{height}
\DeclareDocumentEnvironment{struct}{ O{} O{} m m m }
{
    \newcount\index
    \newcommand{\diaName}{#3}
    \newcommand{\diaWidth}{#4}
    \newcommand{\diaHeight}{#5}
    \newcommand{\updateIndex}{\advance\index by 1\relax}
    \newcommand{\lastId}{\diaName-\the\numexpr\index-1\relax}
    \newcommand{\id}{\diaName-\the\index}
    %% \record[style for node]{node content}
    \newcommand{\record}[2][]{
        \node[fixed node={\diaWidth}{\diaHeight},below=of \lastId, ##1](\id){##2};
        \updateIndex
    }
    %% dummy node for label(title)
    \node[draw,fixed node={\diaWidth}{0em},label={[name=\diaName-label,yshift=0.5em]\textbf{#1}},#2](\id){};\updateIndex
}
{
    %% dummy node for wrapping all nodes in this environment
    \node[fit=(\diaName-0)(\lastId),inner sep=0pt](\diaName){};
}
%%% environment struct end


%%% environment arr: [label][style for dummy node]{name}{width}{height}
\DeclareDocumentEnvironment{arr}{ O{} O{} m O{3em} O{1em} }
{
    \newcount\index
    \newcommand{\diaName}{#3}
    \newcommand{\diaWidth}{#4}
    \newcommand{\diaHeight}{#5}
    \newcommand{\updateIndex}{\advance\index by 1\relax}
    \newcommand{\lastId}{\diaName-\the\numexpr\index-1\relax}
    \newcommand{\id}{\diaName-\the\index}
    %% \element[style for node]{node content}
    \newcommand{\element}[2][]{
        \node[label=below:\tiny{\the\numexpr\index-1\relax},fixed node={\diaWidth}{\diaHeight},right=of \lastId, ##1](\id){##2};
        \updateIndex
    }
    \newcommand{\elem}[2][]{
        \node[fixed node={\diaWidth}{\diaHeight},right=of \lastId, ##1](\id){##2};
        \updateIndex
    }
    %% dummy node for label(title)
    \node[draw,fixed node={0em}{\diaHeight},label={[name=\diaName-label]180:{\bf #1}},#2](\id){};\updateIndex
}
{
    %% dummy node for wrapping all nodes in this environment
    \node[fit=(\diaName-0)(\lastId),inner sep=0pt](\diaName){};
}
%%% environment struct end


\tikzset{
    table/.style 2 args={
        draw,
        rectangle,
        inner sep=0pt,
        matrix of nodes,
        nodes in empty cells,
        nodes={
            draw,
            font=\ttfamily,
            align=center,
            text width=#1,
            outer sep=0pt,
            inner sep=0.3em,
            minimum height=1.6em
        },
        label={[align=center]90:\bf{#2}}
    },
    table/.default={10cm}{},
    right brace/.style 2 args={
        decorate,
        decoration={brace,amplitude=#1,raise=#2}
    },
    right brace/.default={8pt}{2pt},
    right note/.style={right=#1},
    right note/.default={0.3cm},
    left note/.style={left=#1},
    left note/.default={0.3cm},
    left brace/.style 2 args={
        decorate,
        decoration={brace,amplitude=#1,raise=#2,mirror}
    },
    left brace/.default={8pt}{2pt},
    right addr note/.style={right=#1},
    right addr note/.default={0.5cm},
    left addr note/.style={left=#1},
    left addr note/.default={0.5cm},
    arrow/.style={>=stealth',->},
    fixed node/.style 2 args={
        node distance=0,
        outer sep=0,
        inner sep=0,
        draw=white,
        fill=green!50,
        font=\ttfamily,
        rectangle,
        align=center,
        minimum width=#1,
        minimum height=#2
    },
    title node/.style={rectangle,inner sep=1em},
    blank cell/.style={fill=white,minimum height=1cm},
    non blank cell/.style={fill=cyan!30,minimum height=#1}
}


\sybtitle{Android中的长度单位(\textcolor{red}{dp})}
\sybauthor{孙延宾}
\sybdate{\today}

\begin{document}
\tt % I love Typewriter font.
%%%%%%%% the title page and toc %%%%%%%%%%
\pagestyle{header}
\sybmaketitle
\tableofcontents
\newpage

%%%%%%% the main content %%%%%%%%%
\pagestyle{main}
\setcounter{page}{1}

\section[现有长度单位不可行]{现有长度单位不可行}
首先我们要说明为什么使用物理单位是不行的。

\subsection[物理长度单位]{物理长度单位}
从人(开发者)的视角看,直接使用人类熟悉的长度单位是最简单最直观的,比如使用{\bf 厘米}。
但是这样的话会丧失灵活性,比如一个4.8x4.8cm的按钮,在4.5寸屏幕和6寸屏幕上显示的大小一定
是一样的,毕竟4.8cm在哪里都是一样的、不变的,如果它变了,那就太诡异了不是吗?
这就要求开发者针对不同大小的屏幕物理尺寸设置不同的值,这一定是灾难性的。

使用绝对长度是肯定不行的!

\subsection[使用像素]{使用像素}
可否直接使用像素呢,毕竟所有的尺寸都要转换成像素值才能显示到屏幕上,这是否可行呢?
同样不可行。设想两个大小一样(2x1.5寸)的屏幕像素密度却不一样,一个160px/inch,一个320px/inch,
那么一个48x48px的按钮在两个屏幕上看起来会相差很大(4倍),开发者同样需要根据不同的屏幕密度
设置不同的尺寸,这同样是灾难性的。

绑定屏幕物理尺寸是行不通的,同样,绑定屏幕像素密度也是不可行的!


\section[新的长度单位]{新的长度单位}
既然如此,我们来创建一个新的长度单位,这个长度单位要能克服以上两种长度单位的缺点才行,
\begin{itemize}
  \item 不随屏幕像素密度变化而变化
  \item 可以随屏幕物理尺寸变化而变化
\end{itemize}
新的长度单位叫"dip"(device independent pixel,简称"dp"),它不是一个绝对长度单位(如米、厘米等),
所以“1dp等于多少厘米?”这样的问题是毫无意义的,因为dp只是一个比例、一个倍数。

不同的手机可以设置自己的比例值,比如A手机设置该比例值为"1.0",那么1dp=1px,B手机设置该比例值
为"1.99",那么1dp=1.99px,这就是dp的含义,在系统内部自动将dp根据该比例值转换为px数。

上面的比例比较好理解,但是现实中,android并不是这样定义dp的,它规定,

\begin{center}
    在像素密度为160的屏幕上,1dp=1px。
\end{center}

那么像素密度为320的屏幕上,1dp=2px,依次类推。各手机可以定义自己的屏幕像素密度,于是dp的比例值
可以计算出来:

$$1dp=\frac{DPI}{160}pixel$$

DPI就是屏幕像素密度,在DisplayMetrics类里面,属性densityDpi指的就是它,而density指的就是那个“倍数”,
那DPI是如何设定的呢?

\begin{javacode}
    // DisplayMetrics.java
    private static int getDeviceDensity() {
        // qemu.sf.lcd_density can be used to override ro.sf.lcd_density
        // when running in the emulator, allowing for dynamic configurations.
        // The reason for this is that ro.sf.lcd_density is write-once and is
        // set by the init process when it parses build.prop before anything else.
        return SystemProperties.getInt("qemu.sf.lcd_density",
            SystemProperties.getInt("ro.sf.lcd_density", DENSITY_DEFAULT));
    }
\end{javacode}

注意:要区分DPI跟屏幕的物理密度,DPI是可以随意设置的,而屏幕的物理密度是固定不变的!

***问:
\begin{itemize}
  \item 联想A388t(5寸屏)\\
  width=480, height=854, densityDpi=240, density=1.5, xdpi=239.05882, ydpi=241.01778
  \item 小米1(4寸屏)\\
  width=480, height=854, densityDpi=240, density=1.5, xdpi=254.0, ydpi=271.145
\end{itemize}
这两台手机的分辨率一样,dpi一样,density一样,可为什么一个是4寸屏,一个是5寸屏?

***答:

width, height, xdpi, ydpi, 4寸, 5寸,这些都是固定的,是屏幕的物理属性。但是densityDpi
是可以设置的,完全可以设置为一样的值,它表示1dp=1.5px。另外,同一个48x48dp的按钮在
这两个手机上大小(物理大小,可以用尺子测量的大小)是不一样的,因为两个按钮用同样
数量的pixel表示,但是两个手机屏幕的物理像素密度是不一样的。


\section[让APP自己做主]{让APP自己做主}

APP可用自己决定使用多大的度量值,这些值在DisplayMetrics类里面设置,

\begin{javacode}
    DisplayMetrics metrics = new DisplayMetrics();
    // 获取默认度量值
    getWindowManager().getDefaultDisplay().getMetrics(metrics);
    // 修改
    metrics.density = 1.5f;
    metrics.densityDpi = 240;
    metrics.heightPixels = 1080;
    metrics.widthPixels = 1920;
    metrics.scaledDensity = 1.0f;
    metrics.xdpi = 254.0f;
    metrics.ydpi = 254.0f;
    // 使用新的度量值
    getResources().getDisplayMetrics().setTo(metrics);
\end{javacode}


\section[实用技巧]{实用技巧}

\subsection[标准屏幕密度]{标准屏幕密度}

\begin{tabular}{r|c|c}
  % after \\: \hline or \cline{col1-col2} \cline{col3-col4} ...
  type    & dpi & density \\\hline
  ldpi    & 120 & 0.75 \\
  \textcolor{red}{mdpi}    & \textcolor{red}{160} & \textcolor{red}{1.0} \\
  hdpi    & 240 & 1.5 \\
  xhdpi   & 320 & 2.0 \\
  xxhdpi  & 480 & 3.0 \\
  xxxhdpi & 640 & 4.0 \\
\end{tabular}

\subsection[常见分辨率]{常见分辨率}

\begin{tabular}{r|l}
  \hline
    QVGA	& 240x320  \\
    HVGA	& 320x480  \\
    WVGA	& 480x800  \\
    QWVGA	& 240x400  \\
    720P	& 1280x720 \\
    1080	& 1920x1080\\
    2K		& 2560x1440\\
  \hline
\end{tabular}

\subsection[常用命令]{常用命令}

\begin{description}
  \item[查看屏幕信息:] adb shell dumpsys display
  \item[临时修改屏幕密度:] adb shell wm density <DPI>
  \item[查看电池容量:] adb shell dumpsys batterystats | grep "Capacity"
  \item[查看可用的服务:] adb shell dumpsys | grep "DUMP OF SERVICE"
  \item[计算屏幕物理尺寸:] $\sqrt{(\frac{width}{xdpi})^2 + (\frac{height}{ydpi})^2}$,
  如$\sqrt{(\frac{1080}{365.76})^2 + (\frac{1920}{369.454})^2} = 5.977$(6寸屏)
\end{description}


\section[参考文献]{参考文献}
\begin{itemize}
  \item https://developer.android.com/guide/practices/screens_support.html
\end{itemize}

\end{document}

