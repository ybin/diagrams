
\documentclass[a4paper,11pt]{article}

\input{article_preamble}
\input{config}

\sybtitle{Android中的长度单位(\textcolor{red}{dp})}
\sybauthor{孙延宾}
\sybdate{\today}

\begin{document}
\tt % I love Typewriter font.
%%%%%%%% the title page and toc %%%%%%%%%%
\pagestyle{header}
\sybmaketitle
\tableofcontents
\newpage

%%%%%%% the main content %%%%%%%%%
\pagestyle{main}
\setcounter{page}{1}

\section[现有长度单位不可行]{现有长度单位不可行}
首先我们要说明为什么使用物理单位是不行的。

\subsection[物理长度单位]{物理长度单位}
从人(开发者)的视角看,直接使用人类熟悉的长度单位是最简单最直观的,比如使用{\bf 厘米}。
但是这样的话会丧失灵活性,比如一个4.8x4.8cm的按钮,在4.5寸屏幕和6寸屏幕上显示的大小一定
是一样的,毕竟4.8cm在哪里都是一样的、不变的,如果它变了,那就太诡异了不是吗?
这就要求开发者针对不同大小的屏幕物理尺寸设置不同的值,这一定是灾难性的。

使用绝对长度是肯定不行的!

\subsection[使用像素]{使用像素}
可否直接使用像素呢,毕竟所有的尺寸都要转换成像素值才能显示到屏幕上,这是否可行呢?
同样不可行。设想两个大小一样(2x1.5寸)的屏幕像素密度却不一样,一个160px/inch,一个320px/inch,
那么一个48x48px的按钮在两个屏幕上看起来会相差很大(4倍),开发者同样需要根据不同的屏幕密度
设置不同的尺寸,这同样是灾难性的。

绑定屏幕物理尺寸是行不通的,同样,绑定屏幕像素密度也是不可行的!


\section[新的长度单位]{新的长度单位}
既然如此,我们来创建一个新的长度单位,这个长度单位要能克服以上两种长度单位的缺点才行,
\begin{itemize}
  \item 不随屏幕像素密度变化而变化
  \item 可以随屏幕物理尺寸变化而变化
\end{itemize}
新的长度单位叫"dip"(device independent pixel,简称"dp"),它不是一个绝对长度单位(如米、厘米等),
所以“1dp等于多少厘米?”这样的问题是毫无意义的,因为dp只是一个比例、一个倍数。

不同的手机可以设置自己的比例值,比如A手机设置该比例值为"1.0",那么1dp=1px,B手机设置该比例值
为"1.99",那么1dp=1.99px,这就是dp的含义,在系统内部自动将dp根据该比例值转换为px数。

上面的比例比较好理解,但是现实中,android并不是这样定义dp的,它规定,

\begin{center}
    在像素密度为160的屏幕上,1dp=1px。
\end{center}

那么像素密度为320的屏幕上,1dp=2px,依次类推。各手机可以定义自己的屏幕像素密度,于是dp的比例值
可以计算出来:

$$1dp=\frac{DPI}{160}pixel$$

DPI就是屏幕像素密度,在DisplayMetrics类里面,属性densityDpi指的就是它,而density指的就是那个“倍数”,
那DPI是如何设定的呢?

\begin{javacode}
    // DisplayMetrics.java
    private static int getDeviceDensity() {
        // qemu.sf.lcd_density can be used to override ro.sf.lcd_density
        // when running in the emulator, allowing for dynamic configurations.
        // The reason for this is that ro.sf.lcd_density is write-once and is
        // set by the init process when it parses build.prop before anything else.
        return SystemProperties.getInt("qemu.sf.lcd_density",
            SystemProperties.getInt("ro.sf.lcd_density", DENSITY_DEFAULT));
    }
\end{javacode}

注意:要区分DPI跟屏幕的物理密度,DPI是可以随意设置的,而屏幕的物理密度是固定不变的!

***问:
\begin{itemize}
  \item 联想A388t(5寸屏)\\
  width=480, height=854, densityDpi=240, density=1.5, xdpi=239.05882, ydpi=241.01778
  \item 小米1(4寸屏)\\
  width=480, height=854, densityDpi=240, density=1.5, xdpi=254.0, ydpi=271.145
\end{itemize}
这两台手机的分辨率一样,dpi一样,density一样,可为什么一个是4寸屏,一个是5寸屏?

***答:

width, height, xdpi, ydpi, 4寸, 5寸,这些都是固定的,是屏幕的物理属性。但是densityDpi
是可以设置的,完全可以设置为一样的值,它表示1dp=1.5px。另外,同一个48x48dp的按钮在
这两个手机上大小(物理大小,可以用尺子测量的大小)是不一样的,因为两个按钮用同样
数量的pixel表示,但是两个手机屏幕的物理像素密度是不一样的。


\section[让APP自己做主]{让APP自己做主}

APP可用自己决定使用多大的度量值,这些值在DisplayMetrics类里面设置,

\begin{javacode}
    DisplayMetrics metrics = new DisplayMetrics();
    // 获取默认度量值
    getWindowManager().getDefaultDisplay().getMetrics(metrics);
    // 修改
    metrics.density = 1.5f;
    metrics.densityDpi = 240;
    metrics.heightPixels = 1080;
    metrics.widthPixels = 1920;
    metrics.scaledDensity = 1.0f;
    metrics.xdpi = 254.0f;
    metrics.ydpi = 254.0f;
    // 使用新的度量值
    getResources().getDisplayMetrics().setTo(metrics);
\end{javacode}


\section[实用技巧]{实用技巧}

\subsection[标准屏幕密度]{标准屏幕密度}

\begin{tabular}{r|c|c}
  % after \\: \hline or \cline{col1-col2} \cline{col3-col4} ...
  type    & dpi & density \\\hline
  ldpi    & 120 & 0.75 \\
  \textcolor{red}{mdpi}    & \textcolor{red}{160} & \textcolor{red}{1.0} \\
  hdpi    & 240 & 1.5 \\
  xhdpi   & 320 & 2.0 \\
  xxhdpi  & 480 & 3.0 \\
  xxxhdpi & 640 & 4.0 \\
\end{tabular}

\subsection[常见分辨率]{常见分辨率}

\begin{tabular}{r|l}
  \hline
    QVGA	& 240x320  \\
    HVGA	& 320x480  \\
    WVGA	& 480x800  \\
    QWVGA	& 240x400  \\
    720P	& 1280x720 \\
    1080	& 1920x1080\\
    2K		& 2560x1440\\
  \hline
\end{tabular}

\subsection[常用命令]{常用命令}

\begin{description}
  \item[查看屏幕信息:] adb shell dumpsys display
  \item[临时修改屏幕密度:] adb shell wm density <DPI>
  \item[查看电池容量:] adb shell dumpsys batterystats | grep "Capacity"
  \item[查看可用的服务:] adb shell dumpsys | grep "DUMP OF SERVICE"
  \item[计算屏幕物理尺寸:] $\sqrt{(\frac{width}{xdpi})^2 + (\frac{height}{ydpi})^2}$,
  如$\sqrt{(\frac{1080}{365.76})^2 + (\frac{1920}{369.454})^2} = 5.977$(6寸屏)
\end{description}


\section[参考文献]{参考文献}
\begin{itemize}
  \item https://developer.android.com/guide/practices/screens_support.html
\end{itemize}

\end{document}

