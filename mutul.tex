
\documentclass[a4paper,11pt]{article}


%%% fontenc
%\usepackage{fontspec,xunicode,xltxtra}
%\setmainfont{Times New Roman}
%\setsansfont{Source Sans Pro}
%\setmonofont{Source Sans Pro}

%%% xeCJK
\usepackage{xeCJK}
%\setCJKmainfont[BoldFont={Adobe Heiti Std},ItalicFont={方正隶变简体}]{Adobe Song Std}
%\setCJKsansfont[BoldFont={Adobe Heiti Std},ItalicFont={方正隶变简体}]{Adobe Song Std}
%\setCJKmonofont+[BoldFont={Adobe Heiti Std},ItalicFont={方正隶变简体}]{Adobe Song Std}
\setCJKmainfont[BoldFont={微软雅黑},ItalicFont={微软雅黑}]{微软雅黑}
\setCJKsansfont[BoldFont={微软雅黑},ItalicFont={微软雅黑}]{微软雅黑}
\setCJKmonofont[BoldFont={微软雅黑},ItalicFont={微软雅黑}]{微软雅黑}
\XeTeXlinebreaklocale "zh"
\XeTeXlinebreakskip=0pt plus 1pt minus 0.1pt

\usepackage{xcolor}
\usepackage{graphicx}

%%% get total page number
\usepackage{lastpage}

%%% customized definition
\makeatletter
\def\sybtitle#1{\def\@sybtitle{#1}}
\def\sybauthor#1{\def\@sybauthor{#1}}
\def\sybdate#1{\def\@sybdate{#1}}
\sybtitle{}
\sybauthor{}
\sybdate{}
\def\sybmaketitle{
  \begin{center}
  \vspace*{.8in}
  {\huge\bfseries\@sybtitle}
  \par
  \vspace{.8in}
  {\Large\@sybauthor}
  \par
  \vspace{.2in}
  \@sybdate
  \vspace{.5in}
  \end{center}
}
\makeatother
\setlength{\parindent}{0pt}
\renewcommand{\today}{\number\month 月 \number\day 日, ~\number\year 年}
\def\lt{\textless}
\def\gt{\textgreater}
\renewcommand\contentsname{\bfseries 目~~录}
\newcommand\bs{\texttt{\symbol{'134}}} % input backslash sign
%\newcommand\bs{\string\} % same as above definition
\long\def\cmd#1{\par\vspace{.5em}\hspace*{2em}#1\vspace{.5em}\par}
\def\cstr#1{\texttt{\string#1}} % e.g. \cstr{\latex}
\long\def\runcode#1{\par\bigskip#1\bigskip\par}
% 我不想看到那么多的underful hbox,尤其是minted环境加上背景色之后
\hbadness=10000
% 适当放宽overful hbox的限制,运行2pt的溢出
\hfuzz=2pt
\parskip=3\lineskip


%%% change background color & add frame for enumerate enviroment
\usepackage{mdframed}
\newmdenv[backgroundcolor=blue!10,linewidth=0pt]{coloredframe}
\newenvironment{coloredenumerate}{
  \begin{coloredframe}
  \begin{enumerate}
}{
  \end{enumerate}
  \end{coloredframe}
}

%%% geometry
\usepackage[includehead,includefoot,hmargin=21mm,vmargin=10.5mm,
            headsep=12pt,headheight=25pt]{geometry}
%\usepackage[includehead,includefoot,hmargin=1.2in,vmargin=1in]{geometry}

%%% fancyhdr
\usepackage{fancyhdr}
\makeatletter
\fancypagestyle{main} {
  \fancyhf{} % clear header & footer
  \fancyhead[L]{\bfseries\@sybtitle}
  \fancyhead[R]{\thepage/\pageref*{LastPage}}
  \renewcommand{\headrulewidth}{0.4pt} % header line
  \renewcommand{\footrulewidth}{0pt} % footer line
}
\fancypagestyle{header} {
  \fancyhf{} % clear header & footer
  \fancyfoot[C]{\roman{page}}
  \renewcommand{\headrulewidth}{0pt} % header line
  \renewcommand{\footrulewidth}{0pt} % footer line
}
\makeatother

\usepackage{titlesec}
\titleformat{\part}{\centering\Large\bfseries}{第\,\thepart\,部分}{1em}{}
\titleformat{\section}{\large\bfseries}{\thesection}{1em}{}
\titleformat{\subsection}{\normalsize\bfseries}{\thesubsection}{1em}{}
%\titlespacing*{章节命令}{左边距}{上文距}{下文距}[右边距]
\titlespacing*{\section}{0pt}{2\baselineskip}{\parsep}
\usepackage{titletoc}
%\titlecontents{标题层次}[左间距]{整体格式}{标题序号}{标题内容}{指引线和页码}[下间距]
%\titlecontents{subsection}[4em]{}{\contentslabel{3em}}{}{\titlerule*[1em]{$\cdot$}\contentspage}


\usepackage[colorlinks=true,linkcolor=blue,linktoc=section]{hyperref}

%%% perfect source code display
\usepackage{minted}
%\usemintedstyle{colorful}
\definecolor{srcbg}{rgb}{0.95,0.95,0.95}
\newminted{java}{linenos,tabsize=4,bgcolor=srcbg}
\newminted{xml}{linenos,tabsize=4,bgcolor=srcbg}
\newminted{cpp}{linenos,tabsize=4,bgcolor=srcbg}
\newminted{bash}{linenos,tabsize=4,bgcolor=srcbg}
\newminted{latex}{linenos,tabsize=4,bgcolor=srcbg}
\newminted{scheme}{linenos,tabsize=4,bgcolor=srcbg}
\newminted{javascript}{linenos,tabsize=4,bgcolor=srcbg}
\newminted{sql}{linenos,tabsize=4,bgcolor=srcbg}
\newminted{make}{linenos,tabsize=4,bgcolor=srcbg}

\usepackage{amsmath}



\usepackage{xparse}
\usepackage{underscore}
\usepackage{tikz}
\usetikzlibrary{shapes,arrows,arrows.meta,positioning,calc,backgrounds,matrix,fit,decorations.pathreplacing}

%%% environment struct: [label][style for dummy node]{name}{width}{height}
\DeclareDocumentEnvironment{struct}{ O{} O{} m m m }
{
    \newcount\index
    \newcommand{\diaName}{#3}
    \newcommand{\diaWidth}{#4}
    \newcommand{\diaHeight}{#5}
    \newcommand{\updateIndex}{\advance\index by 1\relax}
    \newcommand{\lastId}{\diaName-\the\numexpr\index-1\relax}
    \newcommand{\id}{\diaName-\the\index}
    %% \record[style for node]{node content}
    \newcommand{\record}[2][]{
        \node[fixed node={\diaWidth}{\diaHeight},below=of \lastId, ##1](\id){##2};
        \updateIndex
    }
    %% dummy node for label(title)
    \node[draw,fixed node={\diaWidth}{0em},label={[name=\diaName-label,yshift=0.5em]\textbf{#1}},#2](\id){};\updateIndex
}
{
    %% dummy node for wrapping all nodes in this environment
    \node[fit=(\diaName-0)(\lastId),inner sep=0pt](\diaName){};
}
%%% environment struct end


%%% environment arr: [label][style for dummy node]{name}{width}{height}
\DeclareDocumentEnvironment{arr}{ O{} O{} m O{3em} O{1em} }
{
    \newcount\index
    \newcommand{\diaName}{#3}
    \newcommand{\diaWidth}{#4}
    \newcommand{\diaHeight}{#5}
    \newcommand{\updateIndex}{\advance\index by 1\relax}
    \newcommand{\lastId}{\diaName-\the\numexpr\index-1\relax}
    \newcommand{\id}{\diaName-\the\index}
    %% \element[style for node]{node content}
    \newcommand{\element}[2][]{
        \node[label=below:\tiny{\the\numexpr\index-1\relax},fixed node={\diaWidth}{\diaHeight},right=of \lastId, ##1](\id){##2};
        \updateIndex
    }
    \newcommand{\elem}[2][]{
        \node[fixed node={\diaWidth}{\diaHeight},right=of \lastId, ##1](\id){##2};
        \updateIndex
    }
    %% dummy node for label(title)
    \node[draw,fixed node={0em}{\diaHeight},label={[name=\diaName-label]180:{\bf #1}},#2](\id){};\updateIndex
}
{
    %% dummy node for wrapping all nodes in this environment
    \node[fit=(\diaName-0)(\lastId),inner sep=0pt](\diaName){};
}
%%% environment struct end


\tikzset{
    table/.style 2 args={
        draw,
        rectangle,
        inner sep=0pt,
        matrix of nodes,
        nodes in empty cells,
        nodes={
            draw,
            font=\ttfamily,
            align=center,
            text width=#1,
            outer sep=0pt,
            inner sep=0.3em,
            minimum height=1.6em
        },
        label={[align=center]90:\bf{#2}}
    },
    table/.default={10cm}{},
    right brace/.style 2 args={
        decorate,
        decoration={brace,amplitude=#1,raise=#2}
    },
    right brace/.default={8pt}{2pt},
    right note/.style={right=#1},
    right note/.default={0.3cm},
    left note/.style={left=#1},
    left note/.default={0.3cm},
    left brace/.style 2 args={
        decorate,
        decoration={brace,amplitude=#1,raise=#2,mirror}
    },
    left brace/.default={8pt}{2pt},
    right addr note/.style={right=#1},
    right addr note/.default={0.5cm},
    left addr note/.style={left=#1},
    left addr note/.default={0.5cm},
    arrow/.style={>=stealth',->},
    fixed node/.style 2 args={
        node distance=0,
        outer sep=0,
        inner sep=0,
        draw=white,
        fill=green!50,
        font=\ttfamily,
        rectangle,
        align=center,
        minimum width=#1,
        minimum height=#2
    },
    title node/.style={rectangle,inner sep=1em},
    blank cell/.style={fill=white,minimum height=1cm},
    non blank cell/.style={fill=cyan!30,minimum height=#1}
}


\sybtitle{Camera APP的互斥实现}
\sybauthor{孙延宾}
\sybdate{\today}

\begin{document}
\tt % I love Typewriter font.
%%%%%%%% the title page and toc %%%%%%%%%%
\pagestyle{header}
\sybmaketitle
\tableofcontents
\newpage

%%%%%%% the main content %%%%%%%%%
\pagestyle{main}
\setcounter{page}{1}

\part[数据结构]{数据结构}
互斥相关的数据结构只有一个:$ListPreference$,不过并非所有属性都与互斥相关,
与互斥相关的只有红色的$mValue$、$mEnable$以及蓝色的三个$list$(list和map)。
\vspace{2em}

\begin{tikzpicture}

\begin{struct}[ListPreference][]{ListPreference}{6cm}{2em}
    \record{String mTitle;}
    \record{CharSequence[] mEntries;}
    \record{int mTitleIcon;}
    \record{int mListIcons[];}

    \record[minimum height=0.2em,fill=black]{}
    \record{String mKey;}
    \record[text=red]{String[] mValue=\{old, mutex\};}
    \record{CharSequence[] mDefaultValues;}
    \record{CharSequence[] mEntryValues;}
    \record{boolean mSave;}
    \record{boolean mLoaded;}
    \record[text=red]{boolean mEnable;}

    \record[minimum height=0.2em,fill=black]{}
    \record[inner sep=0.5em]{ArrayList<String> \\ \textcolor{blue}{mModeLevelList};}
    \record[inner sep=0.5em]{HashMap<String, String> \\ \textcolor{blue}{mFunctionLevelList};}
    \record[inner sep=0.5em]{HashMap<String, String> \\ \textcolor{blue}{mFunctionPriorityList};}

    \record[minimum height=0.2em,fill=black]{}
    \record[inner sep=0.5em]{SharedPreferences \\mSharedPreferences;}
    \record[inner sep=0.5em]{OnPreferenceSettingListener \\mListener;}
\end{struct}

\begin{arr}[][right=3cm of ListPreference-14]{snapmode list}[1.5cm][2em]
    \elem{auto}
    \elem{beauty}
    \elem{pro}
    \elem{...}
    \elem{}
\end{arr}

\begin{struct}[][right=2cm of ListPreference-16]{priolist-0}{2cm}{2em}
    \record{mode-1}
    \record{mode-2}
    \record{mode-3}
    \record[minimum height=0.2em,fill=black]{}
    \record{mode-4}
    \record{mode-5}
    \record[minimum height=0.2em,fill=black]{}
    \record{...}
\end{struct}

\begin{struct}[][right=0.05em of priolist-0-0]{priolist-1}{2cm}{2em}
    \record{key-1}
    \record{key-1}
    \record{key-1}
    \record[minimum height=0.2em,fill=black]{}
    \record{key-2}
    \record{key-2}
    \record[minimum height=0.2em,fill=black]{}
    \record{...}
\end{struct}

\begin{struct}[][right=8cm of ListPreference-15]{funclist-0}{2cm}{2em}
    \record{key-1}
    \record[minimum height=0.2em,fill=black]{}
    \record{key-2}
    \record{key-3}
    \record{key-4}
    \record{key-5}
    \record[minimum height=0.2em,fill=black]{}
    \record{...}
\end{struct}

\begin{struct}[][right=0.05em of funclist-0-0]{funclist-1}{2cm}{2em}
    \record{mode-5}
    \record[minimum height=0.2em,fill=black]{}
    \record{mode-1}
    \record{mode-1}
    \record{mode-1}
    \record{mode-1}
    \record[minimum height=0.2em,fill=black]{}
    \record{...}
\end{struct}

%%% lines
\draw[arrow] (ListPreference-14) -- (snapmode list);
\draw[arrow] (ListPreference-15) -- (funclist-0-0);
\draw[arrow] (ListPreference-16) -- (priolist-0-0);

\end{tikzpicture}


\vspace{2em}
\begin{itemize}
  \item mValue是一个length为2的数组,第一个元素表示其本来的值,第二个元素表示它跟别的功能冲突而被禁用时的临时值(一般为$off$)
  \item mEnable表示互斥的结果,互斥的结果只有两个,要么打开、要么被关闭(有特例)
  \item 蓝色的三个list表示与该preference冲突的功能点,注意功能点是与模式相关的,$auto$模式下的闪光灯跟$professional$模式下的闪光灯是不一样的。
        其中的两个hashmap其实是表达一种“多对多”的关系:一个模式对应多个功能、一个功能也可以对应多个模式。
\end{itemize}


\part[操作流程]{操作流程}
数据结构明确了,那$ListPreference$里的数据是如何修改的呢?答案就是$MutualHandle$,
任何一个preference的打开、关闭,以及与其冲突的功能点设置都是在MutualHandle中操作的。

该类主要做了两件事情:
\begin{enumerate}
  \item 为每一个可能存在冲突的功能点定义一个处理函数,注意这里的功能点与模式无关,它与上面定义的“功能点”不同。
  \item 设置互斥的入口:$doMutualHandle(manager, key)$,即有preference变动就要到互斥里面来走一圈。
\end{enumerate}

下面是一个代码实例,

\begin{javacode}
/*
 * xxxMutual:
 *    当xxx功能(如flash)因为冲突(或取消冲突)而disable(或enable)时,
 *    调用该函数来处理xxx的preference以及处理与该功能存在冲突的其他功能。
 *
 * enable: 打开还是关闭
 * mutualState: 当该功能被关闭时要设置的值,一般为off,表示因为冲突而被关闭
 * levelValue: 模式级别还是功能级别,
 *     1. 模式级别表示在某个模式下该功能enable/disable
 *     2. 功能级别表示该功能与某个功能冲突,注意功能是与模式相关的
 * modeValue: 相关的模式
 * lastMutualKey: 功能点,如flash,与modeValue组成具体的功能点
 * beAssociate: 处理循环冲突。如A、B互相冲突,水火不容,A变动会引起B变动,
 *     但是B变动是就不要再引起A变动了,否则就死循环了。
 */
private void xxxMutual(SettingManager manager, boolean enable, String mutualState,
    int levelValue, String modeValue, String lastMutualKey, boolean beAssociate) {
    // 1. 拿到pref对象,因为要修改的就是它
    ListPreference pref = ((ListPreference)
                SettingProvider.getListPreference(SettingFuncConstant.KEY_XXX));
    if (pref == null)
        return;
    // 2. 把key添加到变动列表里,告知UI哪些preference有变动
    manager.addChangedKey(pref.getKey());
    // 3. 记录与pref冲突的功能点,并enable/disable pref
    pref.setPriority(enable, levelValue, modeValue, lastMutualKey);
    // 4. 如果pref是被disable了,那disable之后要怎么变化呢,
    //    一般是将其值设置为"off",或者0,如倒计时,等等
    if (!pref.isEnable()) {
        pref.setValue(mutualState, ListPreference.UPDATE_VALUE_MUTEX);
    }
    // 5. 避免循环冲突
    if (!beAssociate)
        return;
    // 6. 最后,哪些功能点会与“我”冲突呢,你们该变得也赶紧变吧
    if ("on".equals(pref.getValue())) {
        supLightMutual(manager, false, PRIORITY_LEVEL_FUNCTION, modeValue,
                       SettingFuncConstant.KEY_LIVE_SHOOT);
    } else {
        supLightMutual(manager, true, PRIORITY_LEVEL_FUNCTION, modeValue,
                       SettingFuncConstant.KEY_LIVE_SHOOT);
    }
}
\end{javacode}

\end{document}
