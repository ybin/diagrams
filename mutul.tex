
\documentclass[a4paper,11pt]{article}

\input{article_preamble}
\input{config}

\sybtitle{Camera APP的互斥实现}
\sybauthor{孙延宾}
\sybdate{\today}

\begin{document}
\tt % I love Typewriter font.
%%%%%%%% the title page and toc %%%%%%%%%%
\pagestyle{header}
\sybmaketitle
\tableofcontents
\newpage

%%%%%%% the main content %%%%%%%%%
\pagestyle{main}
\setcounter{page}{1}

\part[数据结构]{数据结构}
互斥相关的数据结构只有一个:$ListPreference$,不过并非所有属性都与互斥相关,
与互斥相关的只有红色的$mValue$、$mEnable$以及蓝色的三个$list$(list和map)。
\vspace{2em}

\begin{tikzpicture}

\begin{struct}[ListPreference][]{ListPreference}{6cm}{2em}
    \record{String mTitle;}
    \record{CharSequence[] mEntries;}
    \record{int mTitleIcon;}
    \record{int mListIcons[];}

    \record[minimum height=0.2em,fill=black]{}
    \record{String mKey;}
    \record[text=red]{String[] mValue=\{old, mutex\};}
    \record{CharSequence[] mDefaultValues;}
    \record{CharSequence[] mEntryValues;}
    \record{boolean mSave;}
    \record{boolean mLoaded;}
    \record[text=red]{boolean mEnable;}

    \record[minimum height=0.2em,fill=black]{}
    \record[inner sep=0.5em]{ArrayList<String> \\ \textcolor{blue}{mModeLevelList};}
    \record[inner sep=0.5em]{HashMap<String, String> \\ \textcolor{blue}{mFunctionLevelList};}
    \record[inner sep=0.5em]{HashMap<String, String> \\ \textcolor{blue}{mFunctionPriorityList};}

    \record[minimum height=0.2em,fill=black]{}
    \record[inner sep=0.5em]{SharedPreferences \\mSharedPreferences;}
    \record[inner sep=0.5em]{OnPreferenceSettingListener \\mListener;}
\end{struct}

\begin{arr}[][right=3cm of ListPreference-14]{snapmode list}[1.5cm][2em]
    \elem{auto}
    \elem{beauty}
    \elem{pro}
    \elem{...}
    \elem{}
\end{arr}

\begin{struct}[][right=2cm of ListPreference-16]{priolist-0}{2cm}{2em}
    \record{mode-1}
    \record{mode-2}
    \record{mode-3}
    \record[minimum height=0.2em,fill=black]{}
    \record{mode-4}
    \record{mode-5}
    \record[minimum height=0.2em,fill=black]{}
    \record{...}
\end{struct}

\begin{struct}[][right=0.05em of priolist-0-0]{priolist-1}{2cm}{2em}
    \record{key-1}
    \record{key-1}
    \record{key-1}
    \record[minimum height=0.2em,fill=black]{}
    \record{key-2}
    \record{key-2}
    \record[minimum height=0.2em,fill=black]{}
    \record{...}
\end{struct}

\begin{struct}[][right=8cm of ListPreference-15]{funclist-0}{2cm}{2em}
    \record{key-1}
    \record[minimum height=0.2em,fill=black]{}
    \record{key-2}
    \record{key-3}
    \record{key-4}
    \record{key-5}
    \record[minimum height=0.2em,fill=black]{}
    \record{...}
\end{struct}

\begin{struct}[][right=0.05em of funclist-0-0]{funclist-1}{2cm}{2em}
    \record{mode-5}
    \record[minimum height=0.2em,fill=black]{}
    \record{mode-1}
    \record{mode-1}
    \record{mode-1}
    \record{mode-1}
    \record[minimum height=0.2em,fill=black]{}
    \record{...}
\end{struct}

%%% lines
\draw[arrow] (ListPreference-14) -- (snapmode list);
\draw[arrow] (ListPreference-15) -- (funclist-0-0);
\draw[arrow] (ListPreference-16) -- (priolist-0-0);

\end{tikzpicture}


\vspace{2em}
\begin{itemize}
  \item mValue是一个length为2的数组,第一个元素表示其本来的值,第二个元素表示它跟别的功能冲突而被禁用时的临时值(一般为$off$)
  \item mEnable表示互斥的结果,互斥的结果只有两个,要么打开、要么被关闭(有特例)
  \item 蓝色的三个list表示与该preference冲突的功能点,注意功能点是与模式相关的,$auto$模式下的闪光灯跟$professional$模式下的闪光灯是不一样的。
        其中的两个hashmap其实是表达一种“多对多”的关系:一个模式对应多个功能、一个功能也可以对应多个模式。
\end{itemize}


\part[操作流程]{操作流程}
数据结构明确了,那$ListPreference$里的数据是如何修改的呢?答案就是$MutualHandle$,
任何一个preference的打开、关闭,以及与其冲突的功能点设置都是在MutualHandle中操作的。

该类主要做了两件事情:
\begin{enumerate}
  \item 为每一个可能存在冲突的功能点定义一个处理函数,注意这里的功能点与模式无关,它与上面定义的“功能点”不同。
  \item 设置互斥的入口:$doMutualHandle(manager, key)$,即有preference变动就要到互斥里面来走一圈。
\end{enumerate}

下面是一个代码实例,

\begin{javacode}
/*
 * xxxMutual:
 *    当xxx功能(如flash)因为冲突(或取消冲突)而disable(或enable)时,
 *    调用该函数来处理xxx的preference以及处理与该功能存在冲突的其他功能。
 *
 * enable: 打开还是关闭
 * mutualState: 当该功能被关闭时要设置的值,一般为off,表示因为冲突而被关闭
 * levelValue: 模式级别还是功能级别,
 *     1. 模式级别表示在某个模式下该功能enable/disable
 *     2. 功能级别表示该功能与某个功能冲突,注意功能是与模式相关的
 * modeValue: 相关的模式
 * lastMutualKey: 功能点,如flash,与modeValue组成具体的功能点
 * beAssociate: 处理循环冲突。如A、B互相冲突,水火不容,A变动会引起B变动,
 *     但是B变动是就不要再引起A变动了,否则就死循环了。
 */
private void xxxMutual(SettingManager manager, boolean enable, String mutualState,
    int levelValue, String modeValue, String lastMutualKey, boolean beAssociate) {
    // 1. 拿到pref对象,因为要修改的就是它
    ListPreference pref = ((ListPreference)
                SettingProvider.getListPreference(SettingFuncConstant.KEY_XXX));
    if (pref == null)
        return;
    // 2. 把key添加到变动列表里,告知UI哪些preference有变动
    manager.addChangedKey(pref.getKey());
    // 3. 记录与pref冲突的功能点,并enable/disable pref
    pref.setPriority(enable, levelValue, modeValue, lastMutualKey);
    // 4. 如果pref是被disable了,那disable之后要怎么变化呢,
    //    一般是将其值设置为"off",或者0,如倒计时,等等
    if (!pref.isEnable()) {
        pref.setValue(mutualState, ListPreference.UPDATE_VALUE_MUTEX);
    }
    // 5. 避免循环冲突
    if (!beAssociate)
        return;
    // 6. 最后,哪些功能点会与“我”冲突呢,你们该变得也赶紧变吧
    if ("on".equals(pref.getValue())) {
        supLightMutual(manager, false, PRIORITY_LEVEL_FUNCTION, modeValue,
                       SettingFuncConstant.KEY_LIVE_SHOOT);
    } else {
        supLightMutual(manager, true, PRIORITY_LEVEL_FUNCTION, modeValue,
                       SettingFuncConstant.KEY_LIVE_SHOOT);
    }
}
\end{javacode}

\end{document}
